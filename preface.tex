


\hspace{2em}本指导书初版写于2022--2023学年春季学期初。在该学期以前,本课程也并非没有实验指导书,只是之前的都是幻灯片形式的,而我为什么想要把它一本书呢?这有两点原因:

\hspace{2em}首先,我要确保我理解了这些实验内容,这样我才能较为自信的为同学们讲解,避免以其昏昏使人昭昭。实际上,这对我而言并不简单,我此前并没有上过这门课程,对机器学习的了解也较为浅薄。为了能当好助教,我要先自学这门课程,至少要先把这些实验内容实验搞明白。然而,以往的幻灯片形式的指导书以图片为主,语言为辅,当缺乏必要的讲解时,以它为主要资料会让基础较为薄弱的我学习起来感到十分痛苦。为此,一方面为了证明我已经掌握了这些实验,另一方面为了让像我一样的缺乏基础的同学更容易理解,我决定用自己的方式重新写一遍这本指导书。

\hspace{2em}其次,由于一个人的能力是有限的,我希望该指导书可以由历届助教以及所有上课的同学们一起参与完成,使得该指导书能够更好的传承、历久弥新。为了让更多的人参与到指导书的编写中来,该指导书的形式或许相较于幻灯片形式更为合适。一方面,使用latex语言,大家可以方便地在github或gitee中写作;另一方面,书本相比幻灯片更成体系,大家可以更加方便地在目录中为自己想要补充的内容找到合适位置。因此,我也在此呼吁选课的各位同学来分享自己的知识。


\hspace{2em}该指导书第一章介绍环境配置,包括本地环境、本地虚拟环境的配置以及华为云、深研院计算资源的使用。第二至五章介绍四次实验内容,以及必要的pytorch中的函数、方法,以及模拟多节点的方法。

\hspace{2em}本实验指导书编写过程中也收到了去年的助教王晓禅同学,以及王智老师实验室内选过该课程的代诗琦、吴鸣洲同学的帮助,在此向他们表示感谢。

\rightline{助教袁新杰\ \ 于2023年3月初}
